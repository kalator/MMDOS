\documentclass[a4paper, 12pt]{article}
\usepackage[total={17cm,25cm}, top=2.5cm, left=2.5cm, right=2.5cm,  includefoot]{geometry}
\usepackage[utf8]{inputenc}
\usepackage{array}
\usepackage{multirow}
\usepackage{hhline}
\usepackage{gensymb}
\usepackage{graphicx}
\graphicspath{ {} }
\usepackage[czech]{babel}
\usepackage{enumitem}
\usepackage{pdfpages}
\usepackage{amsmath}
\usepackage{verbatim}
\usepackage{listings}
\usepackage{hyperref}
\usepackage{amssymb}
\usepackage{hyperref}

\usepackage{listings}
\usepackage{color}
 
\definecolor{codegreen}{rgb}{0,0.6,0}
\definecolor{codegray}{rgb}{0.5,0.5,0.5}
\definecolor{codepurple}{rgb}{0.58,0,0.82}
\definecolor{backcolour}{rgb}{0.95,0.95,0.92}
 
\lstdefinestyle{mystyle}{
    backgroundcolor=\color{backcolour},   
    commentstyle=\color{codegreen},
    keywordstyle=\color{magenta},
    numberstyle=\tiny\color{codegray},
    stringstyle=\color{codepurple},
    basicstyle=\footnotesize,
    breakatwhitespace=false,         
    breaklines=true,                 
    captionpos=b,                    
    keepspaces=true,                 
    numbers=left,                    
    numbersep=5pt,                  
    showspaces=false,                
    showstringspaces=false,
    showtabs=false,                  
    tabsize=2
}
 
\lstset{style=mystyle}


\pagestyle{empty} % vypne číslování stránek




%\usepackage[OT2,OT1]{fontenc}
\newcommand\cyr
{
\renewcommand\rmdefault{wncyr}
\renewcommand\sfdefault{wncyss}
\renewcommand\encodingdefault{OT2}
\normalfont
\selectfont
}
\DeclareTextFontCommand{\textcyr}{\cyr}
\def\cprime{\char"7E }
\def\cdprime{\char"7F }
\def\eoborotnoye{\char’013}
\def\Eoborotnoye{\char’003}

\setlength\parindent{0pt}


\begin{document}



\begin{titlepage}
\begin{center}
\noindent
\Large \textbf{České vysoké učení technické v Praze }\\ Fakulta stavební
\vspace{5cm}

\Large

%vložení loga cvut
%\begin{figure}[h!]
%	\centering
%	\includegraphics[width=7cm]{logo.png}
%\end{figure}

\vspace{4cm}

155UZPD: Semestrální projekt \\
\Huge

\textbf{MMDOS - Síťové analýzy PID}
\vspace{11cm}

\large
Únor 2019 \hspace{5cm} Petra Mariana Millarová, Michael Kala\\

\end{center}

\end{titlepage}




\pagestyle{plain}     % zapne obyčejné číslování
\setcounter{page}{1}  % nastaví čítač stránek znovu od jedné

%\tableofcontents
%\newpage

\section{Cíl projektu}
Cílem semestrálního projektu bylo vytvoření konzolové aplikace MMDOS, jež provádí síťové analýzy nad daty PID za využití pgRouting, extenze PostGISu. Uživatel má možnost zadat výchozí a cílovou adresu, aplikace vyhledá nejbližší zastávky hromadné dopravy, provede síťovou analýzu a vrátí nejkratší cestu. Uživateli se poté zobrazí seznam zastávek a linek, které by měl po cestě využít.

\newpage
\section{Data}
\begin{itemize}
	\item Adresní místa RÚIAN Hlavního města Prahy, staženo z Nahlížení do KN [1] (dokumentace: \url{http://vdp.cuzk.cz/vymenny_format/csv/ad-csv-struktura.pdf})
	\item Síť tras linek Pražské integrované dopravy, staženo z portálu Opendata Hlavního města Prahy [2] (dokumentace: \url{http://www.geoportalpraha.cz/cs/fulltext_geoportal?id=6F576389-385E-4E38-831A-8DE6EFB52C3A#.XErv8stKiV4})
	\item Zastávky PID - jednotlivé označníky, staženo z portálu Opendata Hlavního města Prahy [2] (dokumentace: \url{http://www.geoportalpraha.cz/cs/fulltext_geoportal?id=63EF19FE-C2FB-4FC2-8C2D-EEBB72C6B81A#.XErxJ8tKiV4})
\end{itemize}

\newpage
\section{Zpracování dat}
\subsection{Adresní místa RUIAN}
Data byla stažena ve formátu \texttt{csv}. Následně pomocí jazyka AWK byl v shellu soubor upraven, aby obsahoval pouze potřebná data (resp. byl vytvořen nový soubor):
\begin{lstlisting}[language=awk]
awk -F \; '{print $11";"$13";"$14";"$15";"$17";"$18}' 20181231_OB_554782_ADR.csv > ruian_adr.csv
\end{lstlisting}

Ponechány byly sloupce ulice, číslo domovní, číslo orientační, číslo orientační znak (a, b..) a souřadnice x, y v S-JTSK. 

Dále byla v databázi na geo102 vytvořena a naplněna tabulka \textit{adr} (včetně geometrie) pomocí dávkového souboru \textit{adr.sql} puštěného pomocí příkazu:

\begin{lstlisting}[language=bash]
psql -h geo102.fsv.cvut.cz -d pgis_uzpd -U uzpd18_a -f adr.sql	
\end{lstlisting}

Obsah dávkového souboru:

\begin{lstlisting}[language=sql]
CREATE TABLE adr(
	gid serial NOT NULL,
	ulice VARCHAR(50),
	c_domovni INTEGER,
	c_orientacni INTEGER,
	co_znak VARCHAR(2),
	y REAL,
	x REAL,
	geom geometry);
	
\copy adr(ulice, c_domovni, c_orientacni, co_znak, y, x) FROM '../ruian_adr.csv' DELIMITER ';' CSV HEADER encoding 'windows-1250';

UPDATE adr SET geom = ST_GeomFromText('POINT(-'||y||' -'||x||')',5514);

DELETE FROM adr WHERE geom IS NULL;

\end{lstlisting}

Aby byly názvy ulic importovány správně s diakritikou, bylo třeba nastavit kódování na uvedenou hodnotu \texttt{windows-1250}. Zároveň byla pro potřeby projektu z dat vymazána adresní místa bez geometrie.

\subsection{Data PID}
Data zastávek a tras linek PID byla stažena ve formátu \textit{shp}. Pro import dat do databáze byl využit nástroj PostGISu \textit{shp2pgsql}, jež konvertuje soubory ve formátu \textit{shp} do databázových tabulek. Toto bylo vykonáno pomocí příkazu (v shellu):

\begin{lstlisting}[language=bash]
shp2pgsql -s 5514 -D -I DOP_PID_TRASY_TS_L.shp | psql -h geo102.fsv.cvut.cz -d pgis_uzpd -U uzpd18_a
\end{lstlisting} 
 
A přímo v databázi byl změněn název vytvořené tabulky:

\begin{lstlisting}[language=sql]
ALTER TABLE dop_pid_trasy_ts_l RENAME TO trasy;
\end{lstlisting} 

Pro potřeby našeho projektu nebyly nutné údaje o nočních linkách, proto byly tyto řádky smazány: 
\begin{lstlisting}[language=sql]
DELETE FROM trasy 
WHERE (l_metro_n IS NULL
	OR l_tram_n IS NOT NULL
	OR l_bus_n IS NOT NULL
	OR l_lan_n IS NOT NULL
	OR l_vlak_n IS NOT NULL
	OR l_lod_n IS NOT NULL)
AND l_metro IS NULL
AND l_tram IS NULL
AND l_bus IS NULL
AND l_lan IS NULL 
AND l_vlak IS NULL 
AND l_lod IS NULL;
\end{lstlisting} 

Analogicky se postupovalo u zastávek:

\begin{lstlisting}[language=bash]
shp2pgsql -s 5514 -D -I DOP_PID_ZASTAVKY_TS_B.shp| psql -h geo102.fsv.cvut.cz -d pgis_uzpd -U uzpd18_a
\end{lstlisting} 

\begin{lstlisting}[language=sql]
ALTER TABLE dop_pid_zastavky_ts_b RENAME TO zastavky;
\end{lstlisting} 

U zastávek byly taktéž smazány řádky obsahující pouze údaje o nočních linkách:
\begin{lstlisting}[language=sql]
DELETE FROM zastavky 
WHERE zast_denno = 2;
\end{lstlisting} 

Vzhledem k prapodivnosti sloupce \texttt{zast\_id}, kde se ukázalo, že více zastávek s různými názvy a různým umístěním mají ID stejné, se autoři rozhodli použít sloupec \texttt{zast\_uzel} jako identifikátor zastávky, protože byla hodnota stejná pro všechny položky se stejným názvem zastávky. Sloupec byl přetypován následujícím příkazem: 

\begin{lstlisting}[language=sql]
ALTER TABLE zastavky 
ALTER COLUMN zast_uzel_ TYPE INTEGER 
USING CAST(zast_uzel_ AS INTEGER);
\end{lstlisting} 

\section{Topologie}
\indent Vzhledem ke komplikacím zmiňovaným v předchozí kapitole se autoři rozhodli si vytvořit topologii bez použití funkce \texttt{pgr\_createTopology}. Nejprve byly do tabulky \texttt{trasy} přidány sloupce \texttt{source} a \texttt{target} a vytvořen prostorový index: 
\begin{lstlisting}[language=sql]
ALTER TABLE trasy ADD COLUMN "source" integer;
ALTER TABLE trasy ADD COLUMN "target" integer;
CREATE INDEX ON trasy USING gist(geom);
\end{lstlisting} 

Následně byl vytvořen skript v Pythonu, který pomocí prostorového dotazu k začátku i konci každé linie v tabulce \texttt{trasy} vybral nejbližsí bod z tabulky \texttt{zastavky} a vrátil ID jeho uzlu, které pak bylo dosazeno do sloupce \texttt{source}, případně \texttt{target}. \\
Prostorový dotaz pro počáteční bod: 
\begin{lstlisting}[language=sql]
SELECT DISTINCT ON(t.gid) t.gid, z.zast_uzel_ FROM trasy t, zastavky z WHERE ST_DWithin(ST_StartPoint(ST_LineMerge(t.geom)), z.geom, 500) AND t.zast_id_od = z.zast_id
\end{lstlisting}

A pro koncový bod: 
\begin{lstlisting}[language=sql]
SELECT DISTINCT ON(t.gid) t.gid, z.zast_uzel_ FROM trasy t, zastavky z WHERE ST_DWithin(ST_EndPoint(ST_LineMerge(t.geom)), z.geom, 500) AND t.zast_id_ka = z.zast_id")
\end{lstlisting} 

Následně byla vytvořena tabulka vertexů pomocí funkce \textit{pgr\_createVerticesTable} a sloupec \texttt{the\_geom} přejmenován na \texttt{geom}:
\begin{lstlisting}[language=sql]
SELECT pgr_createVerticesTable('trasy','geom','source','target');
ALTER TABLE trasy_vertices_pgr rename column the_geom to geom;
\end{lstlisting} 

\section{Konzolová aplikace}
Pro potřeby konzolové aplikace byly vytvořeny následující SQL funkce: 
\begin{lstlisting}[language=sql]
CREATE OR REPLACE FUNCTION FindStationName(id INTEGER)
RETURNS VARCHAR AS $name$
declare 
	name varchar;
BEGIN 
	SELECT z.zast_nazev INTO name FROM zastavky z
	WHERE z.zast_uzel_ = id LIMIT 1; 
	RETURN name;
END;
$name$ LANGUAGE plpgsql;
\end{lstlisting} 
Tato funkce vrací název zastávky podle zadaného ID ze \texttt{zast\_uzel\_}.

\begin{lstlisting}[language=sql]
CREATE OR REPLACE FUNCTION FindVertexID(cd INTEGER, co INTEGER, u VARCHAR)
RETURNS INTEGER AS $id$
declare 
	id integer;
BEGIN 
	SELECT v.id INTO id FROM adr a, trasy_vertices_pgr v 
	WHERE a.c_domovni = cd 
	AND a.c_orientacni = co 
	AND a.ulice = u
	ORDER BY (a.geom)<->(v.geom) asc limit 1;
	RETURN id;
END;
$id$ LANGUAGE plpgsql;

CREATE OR REPLACE FUNCTION FindVertexIDcd(cd INTEGER, u VARCHAR)
RETURNS INTEGER AS $id$
declare 
	id integer;
BEGIN 
	SELECT v.id INTO id FROM adr a, trasy_vertices_pgr v 
	WHERE a.c_domovni = cd 
	AND a.ulice = u
	ORDER BY (a.geom)<->(v.geom) asc limit 1;
	RETURN id;
END;
$id$ LANGUAGE plpgsql;

CREATE OR REPLACE FUNCTION FindVertexIDori(co INTEGER, u VARCHAR)
RETURNS INTEGER AS $id$
declare 
	id integer;
BEGIN 
	SELECT v.id INTO id FROM adr a, trasy_vertices_pgr v 
	WHERE a.c_orientacni = co
	AND a.ulice = u
	ORDER BY (a.geom)<->(v.geom) asc limit 1;
	RETURN id;
END;
$id$ LANGUAGE plpgsql;

CREATE OR REPLACE FUNCTION FindVertexIDst(u VARCHAR)
RETURNS TABLE(
id BIGINT,
ul VARCHAR,
cp INTEGER,
co INTEGER
	) as $$
BEGIN 
	RETURN QUERY SELECT v.id, a.ulice, a.c_domovni, a.c_orientacni FROM adr a, trasy_vertices_pgr v 
	WHERE a.ulice = u
	ORDER BY (a.geom)<->(v.geom) asc limit 1;
 END; 
$$
LANGUAGE plpgsql;
\end{lstlisting} 
Tyto funkce vrací ID zastávky ( \texttt{zast\_uzel\_} prostřednictvím vertexové tabulky), která je nejblíže zadanému adresnímu bodu. První tři funkce jsou uzpůsobeny na zadání kombinace čísla orientačního, domovního, nebo obou a názvu ulice a poslední funkce umožňuje uživateli zadat jen název ulice, přičemž funkce vrací i adresní bod, ke kterému bylo hledání nejbližší zastávky vztaženo. \\

Následně byl napsán samotný skript aplikace.

\section{Závěr}
\indent Vytvořili jsme konzolovou aplikaci pro hledání spojů PID, do které lze zadat adresu či její část a aplikace vrací nejkratší trasu včetně linek a přestupů. Námětem na vylepšení je například odlišné ocenění tras metra, tramvají a autobusů, přidání možnosti cestovat v noci, přidání jízdních řádu atp. \\
\indent Aplikace samozřejmě nemůže nahradit veřejné služby jako je iDoS nebo  vyhledávač spojení od Dopravního podniku hlavního města Prahy, do kterých bylo vloženo o mnoho více času a úsilí. 

\section{Zdroje}
\noindent
[1] Nahlížení do KN, aplikace ČÚZK - \texttt{https://nahlizenidokn.cuzk.cz/} 

\noindent
[2] Portál pro Otevřená data hlavního města Prahy - \texttt{http://opendata.praha.eu/}

\noindent
[3] Tvoření topologie pomocí PGRouting - \texttt{http://docs.pgrouting.org/latest/en/pgr\_createTopology.html}

\noindent
[4] Tvorba nové tabulky vertexů - \texttt{http://docs.pgrouting.org/latest/en/pgr\_createVerticesTable.html}

\noindent 
[5] Funkce pro vyhledání nejkratší trasy - \texttt{http://docs.pgrouting.org/latest/en/pgr\_dijkstra.html}
 
\end{document}



 