\documentclass[a4paper, 12pt]{article}
\usepackage[total={17cm,25cm}, top=2.5cm, left=2.5cm, right=2.5cm,  includefoot]{geometry}
\usepackage[utf8]{inputenc}
\usepackage{array}
\usepackage{multirow}
\usepackage{hhline}
\usepackage{gensymb}
\usepackage{graphicx}
\graphicspath{ {} }
\usepackage[czech]{babel}
\usepackage{enumitem}
\usepackage{pdfpages}
\usepackage{amsmath}
\usepackage{verbatim}
\usepackage{listings}
\usepackage{hyperref}
\usepackage{amssymb}
\usepackage{hyperref}

\usepackage{listings}
\usepackage{color}
 
\definecolor{codegreen}{rgb}{0,0.6,0}
\definecolor{codegray}{rgb}{0.5,0.5,0.5}
\definecolor{codepurple}{rgb}{0.58,0,0.82}
\definecolor{backcolour}{rgb}{0.95,0.95,0.92}
 
\lstdefinestyle{mystyle}{
    backgroundcolor=\color{backcolour},   
    commentstyle=\color{codegreen},
    keywordstyle=\color{magenta},
    numberstyle=\tiny\color{codegray},
    stringstyle=\color{codepurple},
    basicstyle=\footnotesize,
    breakatwhitespace=false,         
    breaklines=true,                 
    captionpos=b,                    
    keepspaces=true,                 
    numbers=left,                    
    numbersep=5pt,                  
    showspaces=false,                
    showstringspaces=false,
    showtabs=false,                  
    tabsize=2
}
 
\lstset{style=mystyle}


\pagestyle{empty} % vypne číslování stránek




%\usepackage[OT2,OT1]{fontenc}
\newcommand\cyr
{
\renewcommand\rmdefault{wncyr}
\renewcommand\sfdefault{wncyss}
\renewcommand\encodingdefault{OT2}
\normalfont
\selectfont
}
\DeclareTextFontCommand{\textcyr}{\cyr}
\def\cprime{\char"7E }
\def\cdprime{\char"7F }
\def\eoborotnoye{\char’013}
\def\Eoborotnoye{\char’003}

\setlength\parindent{0pt}


\begin{document}



\begin{titlepage}
\begin{center}
\noindent
\Large \textbf{České vysoké učení technické v Praze }\\ Fakulta stavební
\vspace{5cm}

\Large

%vložení loga cvut
%\begin{figure}[h!]
%	\centering
%	\includegraphics[width=7cm]{logo.png}
%\end{figure}

\vspace{4cm}

155UZPD: Semestrální projekt \\
\Huge

\textbf{MMDOS - Síťové analýzy PID}
\vspace{11cm}

\large
Únor 2019 \hspace{5cm} Petra Mariana Millarová, Michael Kala\\

\end{center}

\end{titlepage}




\pagestyle{plain}     % zapne obyčejné číslování
\setcounter{page}{1}  % nastaví čítač stránek znovu od jedné

%\tableofcontents
%\newpage

\section{Cíl projektu}
Cílem semestrálního projektu bylo vytvoření konzolové aplikace MMDOS, jež provádí síťové analýzy nad daty PID za využití pgRouting, extenze PostGISu. Uživatel má možnost zadat výchozí a cílovou adresu, aplikace vyhledá nejbližší zastávky hromadné dopravy, provede síťovou analýzu a vrátí nejkratší cestu. Uživateli se poté zobrazí seznam zastávek a linek, které by měl po cestě využít.

\newpage
\section{Data}
\begin{itemize}
	\item Adresní místa RÚIAN Hlavního města Prahy, staženo z Nahlížení do KN [1] (dokumentace: \url{http://vdp.cuzk.cz/vymenny_format/csv/ad-csv-struktura.pdf})
	\item Síť tras linek Pražské integrované dopravy, staženo z portálu Opendata Hlavního města Prahy [2] (dokumentace: \url{http://www.geoportalpraha.cz/cs/fulltext_geoportal?id=6F576389-385E-4E38-831A-8DE6EFB52C3A#.XErv8stKiV4})
	\item Zastávky PID - jednotlivé označníky, staženo z portálu Opendata Hlavního města Prahy [2] (dokumentace: \url{http://www.geoportalpraha.cz/cs/fulltext_geoportal?id=63EF19FE-C2FB-4FC2-8C2D-EEBB72C6B81A#.XErxJ8tKiV4})
\end{itemize}

\newpage
\section{Zpracování dat}
\subsection{Adresní místa RUIAN}
Data byla stažena ve formátu \texttt{csv}. Následně pomocí jazyka AWK byl v shellu soubor upraven, aby obsahoval pouze potřebná data (resp. byl vytvořen nový soubor):
\begin{lstlisting}[language=awk]
awk -F \; '{print $11";"$13";"$14";"$15";"$17";"$18}' 20181231_OB_554782_ADR.csv > ruian_adr.csv
\end{lstlisting}

Ponechány byly sloupce ulice, číslo domovní, číslo orientační, číslo orientační znak (a, b..) a souřadnice x, y v S-JTSK. 

Dále byla v databázi na geo102 vytvořena a naplněna tabulka \textit{adr} (včetně geometrie) pomocí dávkového souboru \textit{adr.sql} puštěného pomocí příkazu:

\begin{lstlisting}[language=bash]
psql -h geo102.fsv.cvut.cz -d pgis_uzpd -U uzpd18_a -f adr.sql	
\end{lstlisting}

Obsah dávkového souboru:

\begin{lstlisting}[language=sql]
CREATE TABLE adr(
	gid serial NOT NULL,
	ulice VARCHAR(50),
	c_domovni INTEGER,
	c_orientacni INTEGER,
	co_znak VARCHAR(2),
	y REAL,
	x REAL,
	geom geometry);
	
\copy adr(ulice, c_domovni, c_orientacni, co_znak, y, x) FROM '../ruian_adr.csv' DELIMITER ';' CSV HEADER encoding 'windows-1250';

UPDATE adr SET geom = ST_GeomFromText('POINT(-'||y||' -'||x||')',5514);
\end{lstlisting}

Aby byly názvy ulic importovány správně s diakritikou, bylo třeba nastavit kódování na uvedenou hodnotu \texttt{windows-1250}.

\subsection{Data PID}
Data zastávek a tras linek PID byla stažena ve formátu \textit{shp}. Pro import dat do databáze byl využit nástroj PostGISu \textit{shp2pgsql}, jež konvertuje soubory ve formátu \textit{shp} do databázových tabulek. Toto bylo vykonáno pomocí příkazu (v shellu):

\begin{lstlisting}[language=bash]
shp2pgsql -s 5514 -D -I DOP_PID_TRASY_TS_L.shp | psql -h geo102.fsv.cvut.cz -d pgis_uzpd -U uzpd18_a
\end{lstlisting} 
 
A přímo v databázi byl změněn název vytvořené tabulky:

\begin{lstlisting}[language=sql]
ALTER TABLE dop_pid_trasy_ts_l RENAME TO trasy;
\end{lstlisting} 

Analogicky se postupovalo u zastávek:

\begin{lstlisting}[language=bash]
shp2pgsql -s 5514 -D -I DOP_PID_ZASTAVKY_TS_B.shp| psql -h geo102.fsv.cvut.cz -d pgis_uzpd -U uzpd18_a
\end{lstlisting} 

\begin{lstlisting}[language=sql]
ALTER TABLE dop_pid_zastavky_ts_b RENAME TO zastavky;
\end{lstlisting} 


 
\section{Zdroje}
\noindent
[1] Nahlížení do KN, aplikace ČÚZK - \texttt{https://nahlizenidokn.cuzk.cz/} 

\noindent
[2] Portál pro Otevřená data hlavního města Prahy - \texttt{http://opendata.praha.eu/}

\end{document}



 